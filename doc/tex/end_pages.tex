%conclusion
\section*{ЗАКЛЮЧЕНИЕ}
\addcontentsline{toc}{section}{ЗАКЛЮЧЕНИЕ}

\newpage

%conclusion
\addcontentsline{toc}{section}{СПИСОК ИСПОЛЬЗОВАННЫХ ИСТОЧНИКОВ}
\renewcommand{\refname}{СПИСОК ИСПОЛЬЗОВАННЫХ ИСТОЧНИКОВ}

\begin{thebibliography}{99}
    \bibitem{bib:russkijLopatin} 
    \textit{Лопатин В.~В. и др.}  Русский орфографический словарь: 
    около 180 000 слов / Иванова О.~Е., Лопатин В.~В., Нечаева И.~В., 
    Чельцова Л.~К. Отв. ред. В. В. Лопатин. --- 2-е изд., испр. и доп. --- 
    М.: Институт русского языка имени В.~В. Виноградова РАН, 
    2004. --- 960 с.
    
    \bibitem{bib:isoQR}
    ISO/IEC 18004:2006. Information technology -- Automatic identification 
    and data capture techniques -- QR Code 2005 bar code symbology 
    specification.~--- 2 edition (Monolingual). --- 114 p. 
    
    \bibitem{bib:gostDM}
    ГОСТ Р ИСО/МЭК 16022 --- 2008. Автоматическая идентификация. Кодирование
    штриховое. Спецификация символики Data Matrix. ISO/IEC 16022:2006.
    Information technology -- Automatic identification and data capture
    tehniques -- Data Matrix bar code symbology specification (IDN).~--- 
    М., 2009 --- 125 с.
\end{thebibliography}
