\documentclass[12pt,a4paper]{article}

\usepackage[utf8]{inputenc}
\usepackage[T2A]{fontenc}
\usepackage{cyrtimes}
\usepackage[russian]{babel}
\usepackage{amsmath}
\usepackage{amsfonts}
\usepackage{amssymb}
\usepackage{graphicx}
\usepackage[left=3cm,right=1.5cm,top=2cm,bottom=2cm]{geometry}
\usepackage[titletoc]{appendix}
\usepackage{multicol}
\usepackage{multirow}
\usepackage[labelsep=endash]{caption}

\pdfcompresslevel=9

\begin{document}

Положим $X = \left\{x_1, x_2, ..., x_n\right\} $ - множество узлов в компьютерной сети.

Пусть каждый узел $x_i \in X$ в момент времени $t \in T$ характеризуется \textit{состоянием} $S(x_i, t)$.

\textit{Состоянием} сети $S(X, t)$ будем называть множество состояний её узлов в момент времени $t$
\begin{equation}
S(X, t) = \left\{S(x_i, t) | x_i \in X\right\}
\end{equation}


Будем считать, что узлы взаимодействуют между собой посредством передачи сообщений, используя сетевой протокол. Тогда положим $m(x_i, x_j)$ - управляющая информация от объекта $x_i$ к $x_j$. Назовём \textit{переходом} изменение состояния узла в результате взаимодействия с участием этого узла. 

Введём множества состояний $A$ и $N$ (от. Attack и Normal соответственно).

$A$ - множество состояний узлов, каждое из которых представляет состояние узла после произведения над ним какой-либо компьютерной атаки, или другими словами множество всевозможных опасных состояний узлов

$N$ - множество нормальных состояний узлов.

Состояние сети будем называть \textit{опасным}, если состояние хотя бы одного узла в этой сети принадлежит множеству $A$.

Таким образом для обнаружения атак в такой сети достаточно наблюдать за состояниями узлов этой сети, а точнее за изменением состояний этих узлов. 

В рамках данной работы будем предполагать, что состояния узлов изменяются только в результате взаимодействия узлов между собой (ввиду того, что предметом исследования являются атаки на компьютерные сети). А поэтому можно ввести множества $M_A$ и $M_N$ -- соответственно множества описаний объектов $m(x_i, x_j)$ потоков информации, приводящих узлы в в опасные и нормальные состояния.


Зафиксируем узел сети $x \in X$. 
Пусть в момент времени $t$ произошло взаимодействие узлов $x$ и $y$ в сети, в результате которого на узел $x$ поступила управляющая информация $I$. В ответ на это узел $x$ выполняет действия, которые в дальнейшем будем называть \textit{реакцией} узла и обозначать $R = f(I)$, где $f$ - функция реагирования с областью определения $D(f)$~--\{множество всех возможных входов\}. По сути эта функция реализована в виде механизма работы конкретного узла $x$ сети и вообще говоря может отличаться для разных узлов. Она и реализует смену состояний узла $x \in X$.

Формальная постановка задачи обнаружения атак в компьютерной сети:

Пусть задано множество $M_tr = \left\{m(x_i, x_j)\right\}$ описаний взаимодействий узлов $x_i$ и $x_j$ сети, где $m(x_i, x_j)$ можно описать в виде набора признаков $(x_1, x_2, ..., x_n)$.


Множество $M_tr$ будем называть обучающим множеством.

Про множество $M_{tr}$ известно, что подмножество аномальных взаимодействий 
$M_{tr_A} \in M_{tr}$ по мощности мало сравнимо с мощностью множества 
нормальных взаимодействий $M_{tr_T} \in M_{tr}$ 
(составляет не более 1-1.5 \% от общей мощности множества $M_{tr}$). 


Собственно сама постановка задачи:
$\forall m(x_i, x_j)$ определить $m(x_i, x_j) \in M_A$ 
или $m(x_i, x_j) \in M_N$, т.е. для любого 
взаимодействия $m(x_i, x_j)$ в сети определить, 
является оно аномальным (несущим угрозу) или нормальным.

Это ссылка на первую таблицу \ref{table:training_test}
Это ссылка на вторую таблицу \ref{table:fix_width}
Это ссылка на третью таблицу \ref{table:fix_n}


\begin{table}
	\caption{Test table snippet}
	\label{table:training_test}
	\begin{center}
		\begin{tabular}{|c|c|c|c|}
		\hline

		\hline
		\multicolumn{2}{|l|}{\textbf{\hspace{35pt}Выборка}} & \multicolumn{2}{|l|}{\textbf{\hspace{45pt}Коэффициент}}\\

		\multicolumn{1}{|c|}{\textbf{обучающая}} & \multicolumn{1}{|c|}{\textbf{тестовая}} & \multicolumn{1}{|c|}{\textbf{обнаружения}} & \multicolumn{1}{|c|}{\textbf{ложных срабатываний}} \\
		\hline P10 & P1  & 55.7   \% & 0.99  \% \\
		\hline P10 & P2  & 51.04  \% & 1.58  \% \\
		\hline P10 & P3  & 53.01  \% & 1.67  \% \\
		\hline P10 & P10 & 53.39  \% & 1.04  \% \\
		\hline P2  & P1  & 46.3   \% & 0.46  \% \\
		\hline P2  & P2  & 22.0   \% & 0.70  \% \\
		\hline P2  & P3  & 29.3   \% & 2.35  \% \\
		\hline P2  & P10 & 23.0   \% & 9.83  \% \\
		\hline P1  & P1  & 28.3   \% & 4.5  \% \\
		\hline P1  & P2  & 50.5   \% & 1.26  \% \\
		\hline P1  & P3  & 38.5   \% & 3.45  \% \\
		\hline P1  & P10 & 50.4   \% & 11.37 \% \\
		\hline P3  & P1  & 56.25  \% & 0.3  \% \\
		\hline P3  & P2  & 18.56  \% & 0.6  \% \\
		\hline P3  & P3  & 18.75  \% & 0.74  \% \\
		\hline P3  & P10 & 23.0   \% & 1.31  \% \\
		\hline

		\hline
		\end{tabular}
	\end{center}
\end{table}

Ну тут какой-то тестовый текст

\begin{table}
	\caption{Test table 2}
	\label{table:fix_width}
	\begin{center}
		\begin{tabular}{|c|c|c|c|}
		\hline

		\hline
		\multirow{2}{*}{\textbf{W}} & \multirow{2}{*}{\textbf{N}} & \multicolumn{2}{|l|}{\textbf{\hspace{45pt}Коэффициент}} \\
		& & \multicolumn{1}{|c|}{\textbf{обнаружения}} & \multicolumn{1}{|c|}{\textbf{ложных срабатываний}} \\
		\hline 20 & 15 \% & 35.7 \% & 1.44 \% \\
		\hline 20 & 7 \% & 66.2 \% & 2.7 \% \\
		\hline 20 & 2 \% & 88 \% & 8.14 \% \\
		\hline

		\hline
		\end{tabular}
	\end{center}
\end{table}

Ну тут какой-то тестовый текст

\begin{table}
	\caption{Test table 3}
	\label{table:fix_n}
	\begin{center}
		\begin{tabular}{|c|c|c|c|}
		\hline

		\hline
		\multirow{2}{*}{\textbf{W}} & \multirow{2}{*}{\textbf{N}} & \multicolumn{2}{|l|}{\textbf{\hspace{45pt}Коэффициент}} \\
		& & \multicolumn{1}{|c|}{\textbf{обнаружения}} & \multicolumn{1}{|c|}{\textbf{ложных срабатываний}} \\
		\hline 30 & 15 \% & 28.1  \% & 1.07 \% \\
		\hline 40 & 15 \% & 30.77 \% & 0.84 \% \\
		\hline 60 & 15 \% & 31.9  \% & 0.7  \% \\
		\hline 80 & 15 \% & 22.84 \% & 0.6  \% \\
		\hline

		\hline
		\end{tabular}
	\end{center}
\end{table}

\end{document}
