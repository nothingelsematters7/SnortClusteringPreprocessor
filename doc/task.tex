\documentclass[12pt,a4paper]{article}

\usepackage[utf8]{inputenc}
\usepackage[T2A]{fontenc}
\usepackage{cyrtimes}
\usepackage[russian]{babel}
\usepackage{amsmath}
\usepackage{amsfonts}
\usepackage{amssymb}
\usepackage{graphicx}
\usepackage[left=3cm,right=1.5cm,top=2cm,bottom=2cm]{geometry}
\usepackage[titletoc]{appendix}
\usepackage{multicol}
\usepackage[labelsep=endash]{caption}

\author{Тумашик Игорь Александрович}
\title{Полутоновые матричные штрихкоды}

\pdfcompresslevel=9

\begin{document}

Положим $X = \left\{x_1, x_2, ..., x_n\right\} $ - множество узлов в компьютерной сети.

Пусть каждый узел $x_i \in X$ в момент времени $t \in T$ характеризуется \textit{состоянием} $S(x_i, t)$.

\textit{Состоянием} сети $S(X, t)$ будем называть множество состояний её узлов в момент времени $t$
\begin{equation}
S(X, t) = \left\{S(x_i, t) | x_i \in X\right\}
\end{equation}


Будем считать, что узлы взаимодействуют между собой посредством передачи сообщений, используя сетевой протокол. Тогда положим $m(x_i, x_j)$ - управляющая информация от объекта $x_i$ к $x_j$. Назовём \textit{переходом} изменение состояния узла в результате взаимодействия с участием этого узла. 

Введём множества состояний $A$ и $N$ (от. Attack и Normal соответственно).

$A$ - множество состояний узлов, каждое из которых представляет состояние узла после произведения над ним какой-либо компьютерной атаки, или другими словами множество опасных состояний узлов

$N$ - множество нормальных состояний узлов

Состояние сети будем называть \textit{опасным}, если состояние хотя бы одного узла в этой сети принадлежит множеству $A$.

Таким образом для обнаружения атак в такой сети достаточно наблюдать за состояниями узлов этой сети, а точнее за изменением состояний этих узлов. 

В рамках данной работы будем предполагать, что состояния узлов изменяются только в результате взаимодействия узлов между собой (ввиду того, что предметом исследования являются атаки на компьютерные сети).


Зафиксируем узел сети $x \in X$. 
Пусть в момент времени $t$ произошло взаимодействие узлов $x$ и $y$ в сети, в результате которого на узел $x$ поступила управляющая информация $I$. В ответ на это узел $x$ выполняет действия, которые в дальнейшем будем называть \textit{реакцией} узла и обозначать $R = f(I)$, где $f$ - функция реагирования с областью определения $D(f)$~-~\{множество всех возможных входов\}. По сути эта функция реализована в виде механизма работы конкретного узла $x$ сети и вообще говоря может отличаться для разных узлов. Она и реализует смену состояний узла $x \in X$.

Задачу обнаружения атак в компьютерной сети можно теперь записать в следующем виде:
\begin{center}

$F(X) \rightarrow min$

\end{center}
где $F(X) = \left|A_s\right|$, $A_s = \left\{ x | x \in X , S(x) \in A \right\}$. Т.е. задачи обнаружения атак - задача минимизации числа атакованных узлов в сети
\end{document}