\documentclass[12pt,a4paper]{article}

\usepackage[utf8]{inputenc}
\usepackage[T2A]{fontenc}
\usepackage{cyrtimes}
\usepackage[russian]{babel}
\usepackage{amsmath}
\usepackage{amsfonts}
\usepackage{amssymb}
\usepackage{graphicx}
\usepackage[left=3cm,right=1.5cm,top=2cm,bottom=2cm]{geometry}
\usepackage[titletoc]{appendix}
\usepackage{multicol}
\usepackage[labelsep=endash]{caption}

\pdfcompresslevel=9

\begin{document}


Предположим мы выбрали метрику $M$ и зафиксировали диаметр кластера $W$. 
Пусть $dist(C, d)$, где $C$ - кластер, $d$ - вектор, --- расстояние по метрике
$M$ между вектором $d$ и вектором, представляющим кластер $C$.
Представляющим вектором для кластера является вектор признаков, который определяет
центр кластера. Будем называть такой представляющий вектор \textit{центроидом}.

\begin{enumerate}

\item Инициализировать множество кластеров $S = \emptyset $

\item Взять вектор признаков $d$ из обучающего множества. Если $S$ пусто, то создать
кластер с вектором $d$ в качестве представляющего для этого кластера, и добавить
созданный кластер в множество $S$. В противном случае найти среди кластеров
 множества $S$ ближайший к $d$. Другими словами, найти кластер $C \in S$, такой что 
 $\forall C_1 \in S, dist(C, d) \leq dist(C_1, d)$.
 
\item Если $dist(C, d) \leq W$, то добавить вектор $d$ в кластер $C$. В противном 
случае, $d$ отстоит более чем на расстояние $\geq W$ от каждого кластера из $S$,
и поэтому для вектора $d$ необходимо создать новый кластер: 
$S \longleftarrow S \cup \left\{C_n\right\}$, где $C_n$ -- кластер, для которого
вектор $d$ является представляющим.

\item Повторить шаги 2 и 3 для всех векторов из обучающего множества

\end{enumerate}

\section{ПОМЕТКА КЛАСТЕРОВ}

Зафиксировав метрику $M$, векторы одинакового типа должны находиться 
ближе друг к другу нежели к векторам другого типа. При правильном выборе
диаметра кластера $W$ в результате кластеризации мы получим множество кластеров, 
в каждом из которых находятся векторы только одного типа. Это соотвествует нашему
второму предположению о данных, которое заключается в качественном различии
нормальных векторов и аномальных векторов.

Учитывая, что мы имеем дело с непомеченными данными, мы не располагаем информацией
о типе вектора (нормальный или аномальный) во время обучения. Поэтому необходимо
найти способ определения классификации кластеров на нормальные и аномальные.
Наше первое предположение о данных заключалось в количественном превосходстве
нормальных вектором над аномальными (> 98 \%) в обучающей выборке. Следовательно, 
велика вероятность того, что нормальные кластеры будут содержать много больше векторов, 
чем аномальные. Таким образом, будем помечать некоторый процент $N$ от общего 
числа кластеров нормальными. Остальные кластеры получат метки аномальных.

С таким подходом может возникнуть небольшая проблема, в зависимости от того,
сколько подтипов для нормальных векторов существует в обучающей выборке.
Может существовать много различных типов векторов нормального поведения
в сети. Взять для примера векторы для разных протоколов - ftp, telnet, http, ssh и др.
Каждый из них может иметь свою точку в пространстве признаков, около которых
будут формироваться кластеры. Это может привести к образованию большого
числа таких <<нормальных>> кластеров для каждого нормального поведения в сети.
Каждый из таких кластеров будет иметь мощность относительно малую по сравнению с 
аномальными кластерами. И в результате этого, нормальные кластеры будут ошибочно
отнесены к аномальным. Для предотвращения такой проблемы необходимо обеспечить
количественное превосходство нормальных векторов в обучающей выборке над
аномальными. Тогда высока вероятность того, что каждый тип нормального поведения
будет адекватно представлен кластером по сравнению с кластерами для аномальных
векторов.

\section{ОБНАРУЖЕНИЕ}

После кластеризации векторов обучающей выборки система готова проивзодить
обнаружение атак. Пусть на вход поступает вектор $d$. Классификации происходит
следующим образом:

\begin{enumerate}

\item Конвертировать $d$ на основе статистической информации об обучающей выборке,
из которой были сформированы кластеры (нормализация). Обозначим полученный вектор $d'$

\item Найти кластер, ближайший к $d'$ по метрике $M$ (т.е. такой кластер $C \in S$, что
$\forall C' \in S : dist(C, d') \leq dist(C',d')$)

\item Классифицировать $d'$ согласно типу кластера (нормальный или аномальный)

\end{enumerate}

Другими словами, мы находим кластер, ближайший к вектору $d$ (нормализованному) и
классифицируем его согласно типу этого кластера.


\end{document}